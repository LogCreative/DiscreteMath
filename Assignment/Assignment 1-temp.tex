\documentclass[12pt]{article}
\usepackage{CJKutf8}
\usepackage{amsmath}
\usepackage{geometry}
\usepackage{fancyhdr}
\usepackage{longtable,booktabs}
\usepackage{enumerate}
\usepackage{enumitem}
\usepackage{amsthm}
\setlist[enumerate,1]{font=\bfseries}
\geometry{left=3.0cm,right=2.0cm,top=3.0cm,bottom=3.0cm}

\newenvironment{firstlayer}%
{\begin{list}{}{\renewcommand{\makelabel}[1]{\textbf{##1}.\hfil}
}}
{\end{list}}
\newenvironment{secondlayer}%
{\begin{list}{}{\renewcommand{\makelabel}[1]{(##1)\hfil}
}}
{\end{list}}

\renewcommand{\proofname}{\textbf{证明}}

\title{第 1 次作业}
\author{Log Creative}
\date{March 6, 2020}

\begin{document}

\begin{CJK}{UTF8}{gbsn}

\maketitle

\begin{firstlayer}
  \item[1] 
  \begin{secondlayer}
    \item[2] ``12是质数''是命题,真值为假。
    \item[4] ``\(x+y=2\)''不是命题。因为不能确定真值。
    \item[6] ``结果对吗?''不是命题,它是疑问句。
    \item[8] ``假如明天是星期日,那么学校放假。''是命题,一般情况下是真值(法定节假日与重大事件调休除外)。
  \end{secondlayer}
  \item[4] 
  \begin{secondlayer}
    \item[2] \(\neg ((P\vee Q)\rightarrow (Q\vee P))\)是永假式。
    \begin{proof}
    \begin{align*}
      &P\vee Q=Q\vee P \\
      \Rightarrow \quad& (P\vee Q)\rightarrow (Q\vee P)=1 \\
      \Rightarrow \quad & \neg ((P\vee Q)\rightarrow (Q\vee P))=0
    \end{align*}
    \end{proof}
    \item[4] \((Q\rightarrow R)\rightarrow((P\rightarrow Q)\rightarrow (P\rightarrow R))\)是重言式。
    \begin{longtable}[]{@{}cccccc@{}}
        \toprule
        \(P\) & \(Q\) & \(R\) & \(Q\rightarrow R\) &
        \((P\rightarrow Q)\rightarrow (P\rightarrow R)\) &
        \((Q\rightarrow R)\rightarrow((P\rightarrow Q)\rightarrow (P\rightarrow R))\)\tabularnewline
        \midrule
        \endhead
        T & T & T & T & T & T\tabularnewline
        T & T & F & F & F & T\tabularnewline
        T & F & T & T & T & T\tabularnewline
        T & F & F & T & T & T\tabularnewline
        F & T & T & T & T & T\tabularnewline
        F & T & F & F & T & T\tabularnewline
        F & F & T & T & T & T\tabularnewline
        F & F & F & T & T & T\tabularnewline
        \bottomrule
    \end{longtable}
    \item[6] \((P\wedge Q)\rightarrow (P\vee Q)\)是重言式。

        \begin{longtable}[]{@{}ccccc@{}}
        \toprule
        \(P\) & \(Q\) & \(P\wedge Q\) & \(P\vee Q\) &
        \((P\wedge Q)\rightarrow (P\vee Q)\)\tabularnewline
        \midrule
        \endhead
        T & T & T & T & T\tabularnewline
        T & F & F & T& T\tabularnewline
        F & T & F & T& T\tabularnewline
        F & F & F & F& T\tabularnewline
        \bottomrule
        \end{longtable}
  \end{secondlayer}
  \item[5] 
  \begin{secondlayer}
    \item[2] \(R\):他个子高但不很胖。\\
        \(P\):他个子高。\\
        \(Q\):他很胖。\\
        则\(R=P\wedge \neg Q\)。
    \item[4] \(R\):他个子不高也不胖。\\
        \(P\):他个子高。\\
        \(Q\):他胖。\\
        则\(R=\neg P\wedge \neg Q\)。
    \item[6] \(R\):他个子矮或他不很胖都是不对的。\\
        \(P\):他个子矮。\\
        \(Q\):他很胖。\\
        则\(R=\neg(P \wedge \neg Q\))。
    \item[8] \(R\):如果嫦娥是虚构的,而如果圣诞老人也是虚构的,那么许多孩子受骗了。\\
        \(P\):嫦娥是虚构的\\
        \(Q\):圣诞老人也是虚构的。\\
        \(S\):许多孩子受骗了。\\
        则\(R=((P\wedge Q)\rightarrow S)\)。
  \end{secondlayer}
\end{firstlayer}


\end{CJK}

\end{document}

