%md->tex->template
\documentclass[12pt]{article}
\usepackage{CJKutf8}
\usepackage{amsmath}
\usepackage{geometry}
\usepackage{fancyhdr}
\usepackage{longtable,booktabs}
\usepackage{enumerate}
\usepackage{enumitem}
\usepackage{amsthm}
\usepackage{amssymb}
\setlist[enumerate,1]{font=\bfseries}
\geometry{left=3.0cm,right=2.0cm,top=3.0cm,bottom=3.0cm}

\newenvironment{firstlayer}%
{\begin{list}{}{\renewcommand{\makelabel}[1]{\textbf{##1}.\hfil}
}}
{\end{list}}
\newenvironment{secondlayer}%
{\begin{list}{}{\renewcommand{\makelabel}[1]{(##1)\hfil}
}}
{\end{list}}

\renewcommand{\proofname}{\textbf{证明}}

\providecommand{\sol}{\textbf{解}.~}

\title{第 5 次作业}
\author{Log Creative}
\date{April 4, 2020}
\begin{document}

\begin{CJK}{UTF8}{gbsn}

\maketitle

\begin{firstlayer}
  \item[5]将下列语句符号化。
  \begin{secondlayer}
    \item[2]凡有理数都可写成分数。
    
    \sol $P(x)$表示“$x$是有理数”,$Q(x)$表示是“$x$是分数”,则
$$
(\forall x)(P(x)\rightarrow Q(x))
$$
    \item[6]凡实数都能比较大小。
    
    \sol $P(x)$表示“$x$是实数”,$Q(x,y)$表示“$x,y$可以比较大小”,则
$$
(\forall x)(\forall y)(P(x)\wedge P(y)\rightarrow Q(x,y))
$$
  \end{secondlayer}
  \item[7]设个体域为$\{ a,b,c\}$,试将下列公式写成命题逻辑公式。
  \begin{secondlayer}
    \item[10]$(\forall y)((\exists x)P(x,y)\rightarrow (\forall x)Q(x,y))$
    
    \sol 
    \begin{align*}
      (\forall y)&((\exists x)P(x,y)\rightarrow (\forall x)Q(x,y))\\
      =&(\forall y)(P(a,y)\vee P(b,y)\vee P(c,y)\rightarrow Q(a,y)\wedge Q(b,y)\wedge Q(c,y))\\
      =&(P(a,a)\vee P(b,a)\vee P(c,a)\rightarrow Q(a,a)\wedge Q(b,a)\wedge Q(c,a))\wedge\\
      &(P(a,b)\vee P(b,b)\vee P(c,b)\rightarrow Q(a,b)\wedge Q(b,b)\wedge Q(c,b))\wedge\\
      &(P(a,c)\vee P(b,c)\vee P(c,c)\rightarrow Q(a,c)\wedge Q(b,c)\wedge Q(c,c))
    \end{align*}
  \end{secondlayer}
  \item[8]判断下列公式是普遍有效的,不可满足的还是可满足的?
  \begin{secondlayer}
    \item[2]$(\exists x)(P(x)\wedge Q(x))\rightarrow((\exists x)P(x)\wedge (\exists x)Q(x))$
    
    \sol 普遍有效。$(\exists x)(P(x)\wedge Q(x))=\text{T}$意味着存在一个$x_0$使得$P(x_0)=\text{T}$而且$Q(x_0)=\text{T}$,也就是$((\exists x)P(x)\wedge (\exists x)Q(x))=\text{T}$,所以普遍有效。
    
    \item[6]$(\forall x)(P(x)\vee \neg P(x))$
    
    \sol
普遍有效。因为$P(x)=\text{T}$时,$\neg P(x)=\text{F}$;$P(x)=\text{F}$时,$\neg P(x)=\text{T}$,所以$P(x)\vee \neg P(x)=\text{T}$,也就是$(\forall x)(P(x)\vee \neg P(x))=\text{T}$,所以普遍有效。
    \item[7]$((\exists x)P(x)\wedge(\exists x)Q(x))\rightarrow (\exists x)(P(x)\wedge Q(x))$
    
    \sol 可满足的。只有满足$P(x_0)=\text{T}$的$x_0$能够使$Q(x_0)=\text{T}$,式子才能成立。
  \end{secondlayer}
  \item[10]设个体域为$\{a,b\}$,并对$P(x,y)$设定为$P(a,a)=\text{T},P(a,b)=\text{F},P(b,a)=\text{F},P(b,b)=\text{T}$计算下列公式的真值。
  \begin{secondlayer}
    \item[1]$(\forall x)(\exists y)P(x,y)=(\forall x)(P(x,a)\vee P(x,b))=\text{T}$
    \item[3]$(\forall x)(\forall y)P(x,y)=(\forall x)(P(x,a)\wedge P(x,b))=\text{F}$
    \item[5]$(\exists y)\neg P(a,y)=\neg P(a,a)\vee\neg P(a,b)=\text{T}$
    \item[7]$(\forall x)(\forall y)(P(x,y)\rightarrow P(y,x))=\text{T}$,由于$(\forall x)(\forall y)(P(a,b)=P(b,a))$
  \end{secondlayer}
\end{firstlayer}


\end{CJK}

\end{document}

