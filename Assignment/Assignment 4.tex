% Options for packages loaded elsewhere
\PassOptionsToPackage{unicode}{hyperref}
\PassOptionsToPackage{hyphens}{url}
%
\documentclass[
]{article}
\usepackage{lmodern}
\usepackage{amssymb,amsmath}
\usepackage{ifxetex,ifluatex}
\ifnum 0\ifxetex 1\fi\ifluatex 1\fi=0 % if pdftex
  \usepackage[T1]{fontenc}
  \usepackage[utf8]{inputenc}
  \usepackage{textcomp} % provide euro and other symbols
\else % if luatex or xetex
  \usepackage{unicode-math}
  \defaultfontfeatures{Scale=MatchLowercase}
  \defaultfontfeatures[\rmfamily]{Ligatures=TeX,Scale=1}
\fi
% Use upquote if available, for straight quotes in verbatim environments
\IfFileExists{upquote.sty}{\usepackage{upquote}}{}
\IfFileExists{microtype.sty}{% use microtype if available
  \usepackage[]{microtype}
  \UseMicrotypeSet[protrusion]{basicmath} % disable protrusion for tt fonts
}{}
\makeatletter
\@ifundefined{KOMAClassName}{% if non-KOMA class
  \IfFileExists{parskip.sty}{%
    \usepackage{parskip}
  }{% else
    \setlength{\parindent}{0pt}
    \setlength{\parskip}{6pt plus 2pt minus 1pt}}
}{% if KOMA class
  \KOMAoptions{parskip=half}}
\makeatother
\usepackage{xcolor}
\IfFileExists{xurl.sty}{\usepackage{xurl}}{} % add URL line breaks if available
\IfFileExists{bookmark.sty}{\usepackage{bookmark}}{\usepackage{hyperref}}
\hypersetup{
  hidelinks,
  pdfcreator={LaTeX via pandoc}}
\urlstyle{same} % disable monospaced font for URLs
\setlength{\emergencystretch}{3em} % prevent overfull lines
\providecommand{\tightlist}{%
  \setlength{\itemsep}{0pt}\setlength{\parskip}{0pt}}
\setcounter{secnumdepth}{-\maxdimen} % remove section numbering

\author{}
\date{}

\begin{document}

第二章 \\
7 判断下列推理式是否正确?

(10)
\(((P\wedge Q)\rightarrow R)\wedge ( ( P\vee Q)\rightarrow \neg R)\Rightarrow P\wedge Q\wedge R\)

错误。理由:

\(((P\wedge Q)\rightarrow R)\wedge (( P\vee Q)\rightarrow \neg R)\rightarrow P\wedge Q\wedge R=\neg ((\neg (P\wedge Q)\vee R)\wedge (\neg ( P\vee Q)\vee \neg R))\vee (P\wedge Q\wedge R)=((P\wedge Q)\wedge \neg R)\vee((P\vee Q)\wedge R)\vee(P\wedge Q\wedge R)=(P\wedge Q\wedge \neg R)\vee(P\wedge R)\vee(Q\wedge R)\vee(P\wedge Q\wedge R)=((P\wedge Q)\wedge(\neg R\vee R))\vee(P\wedge R)\vee(Q\wedge R)=(P\wedge Q)\vee (P\wedge R)\vee(Q\wedge R)\)

当\(P=Q=\text{F}\)时,该式为\(\text{F}\)。故根据\(P\rightarrow Q\)为真与\(P\Rightarrow Q\)等价的关系可以得到推理式错误。

(11)
\(P \rightarrow Q \Rightarrow ( P \rightarrow R) \rightarrow (Q \rightarrow R)\)

错误。理由:

\((P \rightarrow Q) \rightarrow (( P \rightarrow R) \rightarrow (Q \rightarrow R))=\neg (\neg P\vee Q)\vee(\neg(\neg P\vee R)\vee (\neg Q\vee R))=(P\wedge \neg Q)\vee(P\wedge \neg R)\vee(\neg Q\vee R)=\neg Q\vee (P\wedge \neg R)\vee R=\neg Q\vee P\vee R\neq \text{T}\)

8 \\
(4)
\(P \vee Q\rightarrow R\wedge S, S\vee E \rightarrow U \Rightarrow P \rightarrow U\)

\(P \vee Q\rightarrow R\wedge S\)

\(P\)

\(R\wedge S\)

\(S\)

\(S\vee E \rightarrow U\)

\(U\)

\(P\rightarrow U\)

(5)
\(\neg R\vee S, S\rightarrow Q, \neg Q\Rightarrow Q \leftrightarrow R\)

\(S\rightarrow Q\)

\(\neg Q\rightarrow \neg S\)

\(\neg Q\)

\(\neg S\)

\(\neg R\vee S\)

\(\neg S \rightarrow \neg R\)

\(\neg R\)

\(Q\leftrightarrow R\)

(6)
\(\neg Q\vee S, ( E\rightarrow \neg U)\rightarrow \neg S\Rightarrow Q\rightarrow E\)

\(\neg Q\vee S\)

\(Q\rightarrow S\)

\(Q\)

\(S\)

\(( E\rightarrow \neg U)\rightarrow \neg S\)

\(S\rightarrow \neg (E\rightarrow \neg U)\)

\(\neg (E\rightarrow \neg U)\)

\(E\wedge U\)

\(E\)

(补充)
\(P\rightarrow (Q\rightarrow R),Q\rightarrow (R\rightarrow S)\Rightarrow P\rightarrow (R\rightarrow S)\)

\(P\rightarrow (Q\rightarrow R)\)

\(P\)

\(Q\rightarrow R\)

\(Q\rightarrow (R\rightarrow S)\)

\((Q\rightarrow R)\rightarrow (R\rightarrow S)\)

\(R\rightarrow S\)

9 证明下列推理关系:

(1) 在大城市球赛中. 如果北京队第三, 那么如果上海队第二, 那么天津队第四.
沈阳队不是第一或北京队第三. 上海队第二. 从而知, 如果沈阳队第一,
那么天津队第四.

证明:

令:\\
\(A_1=\)沈阳队第一,\(A_2=\)上海队第二,\(A_3=\)北京队第三,\(A_4=\)天津队第四。

则原命题可以陈述为:\(A_3\rightarrow (A_2\rightarrow A_4),\neg A_1\vee A_3,A_2\Rightarrow A_1\rightarrow A_4\)

\(\neg A_1\vee A_3\)

\(A_1\rightarrow A_3\)

\(A_1\)

\(A_3\)

\(A_3\rightarrow (A_2\rightarrow A_4)\)

\(A_2\rightarrow A_4\)

\(A_2\)

\(A_4\)

\(A_1\rightarrow A_4\)

12 利用归结法证明

(1)\(( P \vee Q) \wedge ( P \rightarrow R) \wedge (Q\rightarrow R)\Rightarrow R\)

证明

\(( P \vee Q) \wedge ( P \rightarrow R) \wedge (Q\rightarrow R)\wedge \neg R=( P \vee Q) \wedge ( \neg P \vee R) \wedge (\neg Q\vee R)\wedge \neg R\)

建立子句集\\
\(S=\{ P \vee Q, \neg P \vee R, \neg Q\vee R, \neg R \}\)

\(P \vee Q\)

\(\neg P \vee R\)

\(\neg Q\vee R\)

\(\neg R\)

\(Q\vee R\)

\(R\)

\(\square\)

第四章

1 判断下列各式是否合式公式

(1) \(P ( x ) \vee(\forall x )Q( x )\)

不是。同一变量两边辖域不同。

(2) \(( \forall x ) ( P ( x ) \wedge Q( x ) )\)

是。

(4) \(( \exists x ) P ( y , z )\)

是。

(6)
\(( \forall x ) ( P ( x ) \wedge R( x ) ) \rightarrow ( ( \forall x ) P ( x ) \wedge Q( x ) )\)

不是。右侧的同一变量两侧辖域不同。

(8) \(( \exists x ) ( ( \forall y) P ( y) \rightarrow  Q( x , y ) )\)

不是。\(( \forall y) P ( y) \rightarrow  Q( x , y )\)内\(y\)的辖域不同。

(9)
\(( \exist x ) ( \exist y ) ( P ( x , y , z ) \rightarrow S( u , v) )\)

是。

2 作如何的具体设定下列公式方为命题

(3) \(( \forall x ) ( \exists y ) P ( x , f ( y , a ) ) \wedge Q( z)\)

当且仅当\(x\),\(y\),\(a\),\(z\)取为常数,\(f\)是常函数,\(P,Q\)为谓词常量。

3 指出下列公式中的自由变元和约束变元, 并指出各量词的辖域

(2)
\(( \forall x ) ( P ( x ) \wedge ( \exist y )Q( y ) ) \wedge ( ( \forall x ) P ( x ) \rightarrow Q( z) )\)

\(z\)是自由变元,\(x,y\)是约束变元。

\(( \forall x ) ( P ( x ) \wedge ( \exist y )Q( y ) )\),\(P ( x ) \wedge ( \exist y )Q( y )\)是\(x\)的辖域。

\(( \exist y )Q( y )\),\(Q( y )\)是\(y\)的辖域。

\(( \forall x ) P ( x )\),\(P ( x )\)是\(x\)的辖域。

(3)
\(( \forall x ) ( P ( x ) \leftrightarrow Q( x ) ) \wedge ( \exist y ) R( y ) \wedge S( z)\)

\(z\)是自由变元,\(x,y\)是约束变元。

\(( \forall x ) ( P ( x ) \leftrightarrow Q( x ) )\),\(P ( x ) \leftrightarrow Q( x )\)是\(x\)的辖域。

\(( \exist y ) R( y )\),\(R( y )\)是\(y\)的辖域。

4 求下列各式的真值

(2)
\(( \exists x ) ( P\rightarrow Q( x ) ) \wedge R( a )\).论域为\(\{- 2, 1, 2, 3, 5, 6\}\),
\(P\) 表 \(2> 1\), \(Q( x )\)表 \(x \leq 3\), \(R( x )\)
表\(x > 5,a= 3\).

解:\\
\(P=\text{T}\)

\((\exists x)(P\rightarrow Q( x ))=\text{T}\)(因为论域中有满足\(x \leq 3\)的数字,\(\text{T}\rightarrow \text{T}=\text{T}\))

\(R(a)=\text{F}\)(因为\(3\ngeq 5\))

\(( \exists x ) ( P\rightarrow Q( x ) ) \wedge R( a )=\text{F}\)

\end{document}
