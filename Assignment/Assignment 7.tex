%md->tex->template
\documentclass[12pt]{article}
\usepackage{CJKutf8}
\usepackage{amsmath}
\usepackage{geometry}
\usepackage{fancyhdr}
\usepackage{longtable,booktabs}
\usepackage{enumerate}
\usepackage{enumitem}
\usepackage{amsthm}
\usepackage{amssymb}
\setlist[enumerate,1]{font=\bfseries}
\geometry{left=3.0cm,right=2.0cm,top=3.0cm,bottom=3.0cm}

\newenvironment{firstlayer}%
{\begin{list}{}{\renewcommand{\makelabel}[1]{\textbf{##1}.\hfil}
}}
{\end{list}}
\newenvironment{secondlayer}%
{\begin{list}{}{\renewcommand{\makelabel}[1]{(##1)\hfil}
}}
{\end{list}}

\renewcommand{\proofname}{\textbf{证明}}

\providecommand{\sol}{\textbf{解}.~}

\title{第 7 次作业}
\author{Log Creative}
\date{April 18, 2020}
\begin{document}

\begin{CJK}{UTF8}{gbsn}

\maketitle

\begin{firstlayer}
  \item[3]指出系列各推演中的错误,并改正之。
  \begin{secondlayer}
    \item[9]
    \begin{align*}
      (\forall x)(\exists y)P(x,y) & && \\
      \text{有}(\exists y)P(a,y) & &&\text{$y$与$a$有关} \\
      \text{有}P(a,b) & &&\text{并非对任一$b$都成立} \\
      (\forall x)P(x,b) & && \text{存在不能导出全称} \\
      P(b,b) & && \\
      (\forall x)P(x,x) & && \text{存在不能导出全称}
    \end{align*}
    \textbf{改正:}所找到的$y$是依赖于$x$的,$P(x,y)$的成立是有条件的。所以结论不一定成立。
    \item[11]
    \begin{align*}
     (\forall x)(P(x)\rightarrow Q(x)) & && \\
     \text{有}P(c)\rightarrow Q(c) & && \\
     (\exists x)P(x) & && \\
     \text{有}P(c) & &&\text{上文的$c$不一定满足这个性质} \\
     Q(c) & && \\
     (\exists x)Q(x) & &&
    \end{align*}
    \textbf{改正:}
    \begin{align*}
     (\forall x)(P(x)\rightarrow Q(x)) & && \text{前提}\\
     (\exists x)P(x) & && \text{前提} \\
     P(c) & && \text{存在量词消去} \\
     P(c)\rightarrow Q(c) & && \text{全称量词消去} \\
     Q(c) & && \text{分离} \\
     (\exists x)Q(x) & && \text{存在量词引入}
    \end{align*}
    
  \end{secondlayer}
\end{firstlayer}

\end{CJK}

\end{document}

