%md->tex->template
\documentclass[12pt]{article}
\usepackage{CJKutf8}
\usepackage{amsmath}
\usepackage{geometry}
\usepackage{fancyhdr}
\usepackage{longtable,booktabs}
\usepackage{enumerate}
\usepackage{enumitem}
\usepackage{amsthm}
\setlist[enumerate,1]{font=\bfseries}
\geometry{left=3.0cm,right=2.0cm,top=3.0cm,bottom=3.0cm}

\newenvironment{firstlayer}%
{\begin{list}{}{\renewcommand{\makelabel}[1]{\textbf{##1}.\hfil}
}}
{\end{list}}
\newenvironment{secondlayer}%
{\begin{list}{}{\renewcommand{\makelabel}[1]{(##1)\hfil}
}}
{\end{list}}

\renewcommand{\proofname}{\textbf{证明}}

\title{第 3 次作业}
\author{Log Creative}
\date{March 21, 2020}
\begin{document}

\begin{CJK}{UTF8}{gbsn}

\maketitle

\begin{firstlayer}
  \item[2] 由下列真值表,分别从T和F来列写出\(A\),\(B\)和\(C\)的表达式,并分别以符号\(m_i\)和\(M_i\)表示.
\begin{longtable}[]{@{}lllll@{}}
\toprule
\(P\) & \(Q\) & \(A\) & \(B\) & \(C\)\tabularnewline
\midrule
\endhead
F & F & T & T & T\tabularnewline
F & T & T & F & F\tabularnewline
T & F & T & F & F\tabularnewline
T & T & F & T & F\tabularnewline
\bottomrule
\end{longtable}

\textbf{解}.

\begin{align}
  A&=P\uparrow Q && \text{(表达式)} \nonumber \\
  &=\neg P \vee \neg Q \nonumber \\
  &=M_0 && (M_i) \nonumber \\
  &=\neg(P\wedge Q) \nonumber \\
  &=\neg m_3 \nonumber \\
  &=\vee_{0,1,2} \nonumber \\
  &=m_0\vee m_1 \vee m_2 && (m_i) 
\end{align}
\begin{align}
  B&=P\leftrightarrow Q && \text{(表达式)} \nonumber\\
  &=(\neg P \wedge \neg Q)\vee (P \wedge Q)  \nonumber \\
  &=m_0\vee m_3 && (m_i) \nonumber \\
  &=\vee_{0,3}\nonumber \\
  &=(P \vee \neg Q)\wedge (\neg P\vee Q) \nonumber \\
  &=M_1\wedge M_2 && (M_i) \\
  &=\wedge_{1,2} \nonumber 
\end{align}
\begin{align}
  C&=P\downarrow Q && \text{(表达式)} \nonumber\\
  &=\neg P\wedge \neg Q \nonumber \\
  &=m_0 && (m_i) \nonumber \\
  &=\neg (P\vee Q) \nonumber \\
  &=\neg M_3 \nonumber \\
  &=M_0\wedge M_1 \wedge M_2 && (M_i) \\
  &=\wedge_{0,1,2} \nonumber 
\end{align}

\item[4]证明
\begin{secondlayer}
  \item[1]\(A\rightarrow B\)与\(B^*\rightarrow A^*\)同永真、同可满足。
  \begin{proof}
    通过证明重言蕴含来同时证明两者。

若\(A\rightarrow B\)真,则\(\neg B\rightarrow \neg A\)真。(逆否命题)

而\(\neg A=A^{*-},\neg B=B^{*-}\),则\(B^{*-}\rightarrow A^{*-}\)真。(置换规则)

假设\(A=A(P_1,P_2,\cdots,P_n)\),\\
则\(A^-=A(\neg P_1,\neg P_2,\cdots,\neg P_n)\),其中\(P_i(i=1,2,\cdots,n)\)是任意的命题,可以将\(\neg P_i\)代入后者,将得到\(A\),也就意味着\(B^*\rightarrow A^*\)真。也就是
\begin{equation}\label{11}
  A\rightarrow B\Rightarrow B^*\rightarrow A^*
\end{equation}
如果\(B^*\rightarrow A^*\)真,则\((A^*)^*\rightarrow (B^*)^*\)永真。(公式 (\ref{11}))

则\(A\rightarrow B\)真。(\((A^*)^*=A\))

也就是
\begin{equation}\label{12}
  B^*\rightarrow A^* \Rightarrow A\rightarrow B
\end{equation}

综合公式 (\ref{11}) 和 (\ref{12}),有
\begin{equation}
  A\rightarrow B\Leftrightarrow B^*\rightarrow A^*
\end{equation}
\begin{secondlayer}
  \item[i 同永真]如果\(A→B\)永真,则根据永真蕴含的定义,\(B^*→A^*\)也将永真。反之亦然。
  \item[ii 同可满足]如果\(A(a_1,a_2,\cdots,a_n),B(b_1,b_2,\cdots,b_n)\)是满足\(A\rightarrow B\)真的一个条件,则根据永真蕴含的定义,\(B^*\rightarrow A^*\)也将真。反之亦然。
\end{secondlayer}

  \end{proof}
  
\item[2]\(A\leftrightarrow B\)与\(A^*\leftrightarrow B^*\)同永真、同可满足。
\begin{proof}
  同样地,若\(A\leftrightarrow B\)真,则\(A=B\)(等值定理),\(\neg A=\neg B\)(同真假),\(\neg A\leftrightarrow \neg B\)真(等值定理)。

之后,根据类似的推理,\(A^{*-}\leftrightarrow B^{*-},A^{*}\leftrightarrow B^{*}\),也就是
\begin{equation}\label{21}
  A\leftrightarrow B \Rightarrow A^*\leftrightarrow B^*
\end{equation}

反之,\(A^*\leftrightarrow B^*\)真,则\((A^*)^*\leftrightarrow (B^*)^*\)真,\(A\leftrightarrow B\)真,也就是
\begin{equation}\label{22}
  A^*\leftrightarrow B^* \Rightarrow A\leftrightarrow B
\end{equation}

综合公式 (\ref{21}) 和 (\ref{22}),有
\begin{equation}
  A\leftrightarrow B \Leftrightarrow A^*\leftrightarrow B^*
\end{equation}

故两者同永真,同可满足。
\end{proof}
\end{secondlayer}
\item[5]给出下列各公式的合取范式、析取范式、主合取范式和主析取范式。并给出所有使公式为真的解释。
\begin{secondlayer}
  \item[4]\((P\wedge Q)\vee (\neg P\wedge Q \wedge R)\)
  
  \textbf{解}.
  %\begin{subequations}
    \begin{align}
    (P\wedge Q)&\vee (\neg P\wedge Q \wedge R) &&\text{(析取范式)} \\
    =&(P\vee \neg P)\wedge (P \vee Q) \wedge (P \vee R) \wedge (Q\vee \neg P)\nonumber\\
    & \wedge (Q\vee Q)\wedge (Q\vee R) \nonumber \\
    =&(P\vee Q)\wedge (P\vee R)\wedge (Q\vee \neg P)\wedge Q\wedge (Q\vee R)\nonumber \\
    =&Q\wedge(P\vee R)&&\text{(合取范式)}\\
    (P\wedge Q)&\vee (\neg P\wedge Q \wedge R)\nonumber\\
    =&(P\wedge Q\wedge (R\vee \neg R))\vee (\neg P\wedge Q \wedge R) \nonumber \\
    =&(P\wedge Q\wedge R)\vee(P\wedge Q\wedge \neg R)\vee (\neg P\wedge Q \wedge R) \nonumber \\
    =&m_7\vee m_6\vee m_3\nonumber \\
    =&\vee_{3,6,7}&&\text{(主析取范式)}\\
    =&\wedge_{\{0,1,2,4,5\}\text{补}} \nonumber \\
    =&\wedge_{2,3,5,6,7}&&\text{(主合取范式)}
  \end{align}
  %\end{subequations}
  使公式为真的解释:

\begin{longtable}[]{@{}llll@{}}
\toprule
\(P\) & \(Q\) & \(R\) & \(m_i\)\tabularnewline
\midrule
\endhead
F & T & T & \(m_3\)\tabularnewline
T & T & F & \(m_6\)\tabularnewline
T & T & T & \(m_7\)\tabularnewline
\bottomrule
\end{longtable}
\newpage
\item[8] $(P\rightarrow Q)\vee ((Q\wedge P)\leftrightarrow (Q\leftrightarrow \neg P))$

  \textbf{解}.
%\begin{subequations}
  \begin{align}
    (P\rightarrow Q)&\vee ((Q\wedge P)\leftrightarrow (Q\leftrightarrow \neg P)) \nonumber \\
    =&(P\rightarrow Q)\vee ((Q\wedge P)\leftrightarrow (\neg Q \wedge \neg \neg P)\vee (Q \wedge \neg P))\nonumber \\
    =&(\neg P \vee Q)\vee [(Q\wedge P)\leftrightarrow (\neg Q \wedge P)\vee (Q \wedge \neg P)]\nonumber \\
    =&(\neg P \vee Q)\vee\{[(Q\wedge P)\wedge ((\neg Q \wedge P)\vee (Q \wedge \neg P))] \nonumber \\ 
    &\vee [\neg (Q\wedge P)\wedge \neg ((\neg Q \wedge P)\vee (Q \wedge \neg P))] \}\nonumber \\
    =&(\neg P \vee Q)\vee\{[((Q\wedge P)\wedge (\neg Q \wedge P))\vee ((Q\wedge P)\wedge\nonumber \\
    &(Q \wedge \neg P))]\vee [\neg (Q\wedge P)\wedge (\neg(\neg Q \wedge P)\wedge \neg(Q \wedge \neg P))] \}\nonumber \\
    =&(\neg P \vee Q)\vee\{\text{F}\vee[(\neg Q\vee \neg P)\wedge (( Q \vee \neg P)\wedge ( P\vee \neg Q))] \}\nonumber \\
    =&(\neg P \vee Q)\vee\{[(\neg Q\vee \neg P)\wedge ( P\vee \neg Q)]\wedge ( \neg P \vee Q) \}\nonumber 
  \end{align}
  \begin{align}
    =&\neg P \vee Q&&\text{(析取范式、合取范式)}\\
    =&M_1\nonumber \\
    =&\wedge_1&& \text{(主合取范式)}\\
    =&\vee_{\{0,2,3\}\text{补}}\nonumber \\
    =&\vee_{0,1,3}&& \text{(主析取范式)}
  \end{align}
%\end{subequations}
使公式为真的解释:

\begin{longtable}[]{@{}lll@{}}
\toprule
\(P\) & \(Q\) & \(m_i\)\tabularnewline
\midrule
\endhead
F & F & \(m_0\)\tabularnewline
F & T & \(m_1\)\tabularnewline
T & T & \(m_3\)\tabularnewline
\bottomrule
\end{longtable}
\end{secondlayer}

\end{firstlayer}

\end{CJK}

\end{document}

