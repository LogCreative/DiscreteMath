% Options for packages loaded elsewhere
\PassOptionsToPackage{unicode}{hyperref}
\PassOptionsToPackage{hyphens}{url}
%
\documentclass[
]{article}
\usepackage{lmodern}
\usepackage{amssymb,amsmath}
\usepackage{ifxetex,ifluatex}
\ifnum 0\ifxetex 1\fi\ifluatex 1\fi=0 % if pdftex
  \usepackage[T1]{fontenc}
  \usepackage[utf8]{inputenc}
  \usepackage{textcomp} % provide euro and other symbols
\else % if luatex or xetex
  \usepackage{unicode-math}
  \defaultfontfeatures{Scale=MatchLowercase}
  \defaultfontfeatures[\rmfamily]{Ligatures=TeX,Scale=1}
\fi
% Use upquote if available, for straight quotes in verbatim environments
\IfFileExists{upquote.sty}{\usepackage{upquote}}{}
\IfFileExists{microtype.sty}{% use microtype if available
  \usepackage[]{microtype}
  \UseMicrotypeSet[protrusion]{basicmath} % disable protrusion for tt fonts
}{}
\makeatletter
\@ifundefined{KOMAClassName}{% if non-KOMA class
  \IfFileExists{parskip.sty}{%
    \usepackage{parskip}
  }{% else
    \setlength{\parindent}{0pt}
    \setlength{\parskip}{6pt plus 2pt minus 1pt}}
}{% if KOMA class
  \KOMAoptions{parskip=half}}
\makeatother
\usepackage{xcolor}
\IfFileExists{xurl.sty}{\usepackage{xurl}}{} % add URL line breaks if available
\IfFileExists{bookmark.sty}{\usepackage{bookmark}}{\usepackage{hyperref}}
\hypersetup{
  hidelinks,
  pdfcreator={LaTeX via pandoc}}
\urlstyle{same} % disable monospaced font for URLs
\usepackage{longtable,booktabs}
% Correct order of tables after \paragraph or \subparagraph
\usepackage{etoolbox}
\makeatletter
\patchcmd\longtable{\par}{\if@noskipsec\mbox{}\fi\par}{}{}
\makeatother
% Allow footnotes in longtable head/foot
\IfFileExists{footnotehyper.sty}{\usepackage{footnotehyper}}{\usepackage{footnote}}
\makesavenoteenv{longtable}
\setlength{\emergencystretch}{3em} % prevent overfull lines
\providecommand{\tightlist}{%
  \setlength{\itemsep}{0pt}\setlength{\parskip}{0pt}}
\setcounter{secnumdepth}{-\maxdimen} % remove section numbering

\author{}
\date{}

\begin{document}

\hypertarget{header-n2}{%
\section{第 1 次作业}\label{header-n2}}

\emph{Log Creative}

1.(2)``12是质数''是命题,真值为假。\\
(4)``\(x+y=2\)''不是命题。因为不能确定真值。\\
(6)``结果对吗?''不是命题,它是疑问句。\\
(8)``假如明天是星期日,那么学校放假。''是命题,一般情况下是真值(法定节假日调休除外)。

4.(2)\(\neg ((P\vee Q)\rightarrow (Q\vee P))\)是永假式。证明:\(P\vee Q=Q\vee P\Rightarrow (P\vee Q)\rightarrow (Q\vee P)=1\Rightarrow \neg ((P\vee Q)\rightarrow (Q\vee P))=0\).

(4)
\((Q\rightarrow R)\rightarrow((P\rightarrow Q)\rightarrow (P\rightarrow R))\)是重言式。

\begin{longtable}[]{@{}cccccc@{}}
\toprule
\(P\) & \(Q\) & \(R\) & \((Q\rightarrow R)\) &
\((P\rightarrow Q)\rightarrow (P\rightarrow R)\) &
\((Q\rightarrow R)\rightarrow((P\rightarrow Q)\rightarrow (P\rightarrow R))\)\tabularnewline
\midrule
\endhead
T & T & T & T & T & T\tabularnewline
T & T & F & F & & T\tabularnewline
T & F & T & T & T & T\tabularnewline
T & F & F & T & T & T\tabularnewline
F & T & T & T & T & T\tabularnewline
F & T & F & F & & T\tabularnewline
F & F & T & T & T & T\tabularnewline
F & F & F & T & T & T\tabularnewline
\bottomrule
\end{longtable}

(6)\((P\wedge Q)\rightarrow (P\vee Q)\)是重言式。

\begin{longtable}[]{@{}ccccc@{}}
\toprule
\(P\) & \(Q\) & \(P\wedge Q\) & \(P\vee Q\) &
\((P\wedge Q)\rightarrow (P\vee Q)\)\tabularnewline
\midrule
\endhead
T & T & T & T & T\tabularnewline
T & F & F & & T\tabularnewline
F & T & F & & T\tabularnewline
F & F & F & & T\tabularnewline
\bottomrule
\end{longtable}

5.(2)\(R\):他个子高但不很胖。\textbackslash{}\\
\(P\):他个子高。\textbackslash{}\\
\(Q\):他很胖。\textbackslash{}\\
则\(R=P\wedge \neg Q\)。

(4)\(R\):他个子不高也不胖。\textbackslash{}\\
\(P\):他个子高。\textbackslash{}\\
\(Q\):他胖。\textbackslash{}\\
则\(R=\neg P\wedge \neg Q\)。

(6)\(R\):他个子矮或他不很胖都是不对的。\textbackslash{}\\
\(P\):他个子矮。\textbackslash{}\\
\(Q\):他很胖。\textbackslash{}\\
则\(R=\neg(P \vee \neg Q\))。

(8)\(R\):如果嫦娥是虚构的,而如果圣诞老人也是虚构的,那么许多孩子受骗了。\textbackslash{}\\
\(P\):嫦娥是虚构的\textbackslash{}\\
\(Q\):圣诞老人也是虚构的。\textbackslash{}\\
\(S\):许多孩子受骗了。\textbackslash{}\\
则\(R=((P\wedge Q)\rightarrow S)\)。

\end{document}
