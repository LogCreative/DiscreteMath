%md->tex->template
\documentclass[12pt]{article}
\usepackage{CJKutf8}
\usepackage{amsmath}
\usepackage{geometry}
\usepackage{fancyhdr}
\usepackage{longtable,booktabs}
\usepackage{enumerate}
\usepackage{enumitem}
\usepackage{amsthm}
\usepackage{amssymb}
\usepackage{tikz}
\setlist[enumerate,1]{font=\bfseries}
\geometry{left=3.0cm,right=2.0cm,top=3.0cm,bottom=3.0cm}

\newenvironment{firstlayer}%
{\begin{list}{}{\renewcommand{\makelabel}[1]{\textbf{##1}.\hfil}
}}
{\end{list}}
\newenvironment{secondlayer}%
{\begin{list}{}{\renewcommand{\makelabel}[1]{(##1)\hfil}
}}
{\end{list}}

\renewcommand{\proofname}{\textbf{证明}}

\providecommand{\sol}{\textbf{解}.~}

\title{第 13 次作业}
\author{Log Creative}
\date{June 6, 2020}
\begin{document}

\begin{CJK}{UTF8}{gbsn}

\maketitle

\begin{firstlayer}
  \item[4] 设$f:\mathbf{N}\rightarrow\mathbf{N},f(x)=\begin{cases}
    1 & \text{当}x\text{是奇数,} \\
    \frac{x}{2} & \text{当}x\text{是偶数.} \\
\end{cases}$

求$f(0),f[\{0\}],f[\{0,2,4,6,\cdots\}],f[\{1,3,5,\cdots\}],f^{-1}[\{2\}],f^{-1}[\{3,4\}]$。

\sol \begin{align*}
    f(0)&=0\\
    f[\{0\}]&=\{0\}\\
    f[\{0,2,4,6,\cdots\}]&=\{0,1,2,3,\cdots\}\\
    f[\{1,3,5,\cdots\}]&=\{1\}\\
    f^{-1}[\{2\}]&=\{4\}\\
    f^{-1}[\{3,4\}]&=\{6,8\}
\end{align*}

\item[5] 对下列函数分别确定:
\begin{secondlayer}
  \item[a]是否是满射的、单射的;如果是双射的,写出$f^{-1}$的表达式.
  \item[b]写出函数的象和对给定集合$S$的完全原象.
  \item[c]关系$R=\{\langle x,y \rangle|x,y\in\text{dom}(f)\wedge f(x)=f(y)\}$是$\text{dom}(f)$上的等价关系,一般称为由函数$f$导出的等价关系,求$R$.
\end{secondlayer}
\begin{secondlayer}
  \item[4]$f:\mathbf{N}\rightarrow\mathbf{N}\times\mathbf{N},f(n)=\langle n,n+1 \rangle,S=\{\langle 2,2 \rangle\}$
  
  \sol (a)不是满射,是单射。
(b)$f[\mathbf{N}]=\{\langle n,n+1 \rangle|n\in \mathbf{N}\};f^{-1}(S)=\varnothing$.
(c)$R=I_{\mathbf{N}}$

    \item[5]$f:[0,1]\rightarrow[0,1],f(x)=\frac{2x+1}{4},S=\left[0,\frac{1}{2}\right]$.

    \sol (a)不是满射,是单射。
(b)$f([0,1])=[\frac{1}{4},\frac{3}{4}];f^{-1}(S)=[0,\frac{1}{2}]$
(c)$R=I_{[0,1]}$
\end{secondlayer}
\item[11]对$f:A\rightarrow B$,定义$g:B\rightarrow P(A)$为$g(b)=\{x|x\in A\wedge f(x)\in b\}$.

证明:若$f$是满射的,则$g$是单射的。其逆是否成立?
\begin{proof}
  对$\forall b_1,b_2\in B\wedge b_1\neq b_2,g(b_1)=\{x|x\in A\wedge f(x)\in b_1\};g(b_2)=\{x|x\in A\wedge f(x)\in b_2\}$

若$f$是满射的:$\exists a_1,a_2\in A:f(a_1)=b_1,f(a_2)=b_2$,如果$a_1=a_2$而$b_1\neq b_2$,那么$f$将不是函数,所以$a_1\neq a_2$。所以$g(b_1)\neq g(b_2)$,$g$是单射的。

逆不成立。对$\forall b\in B$,$g(b)$只能保证是独特的,并且$g(b)=\varnothing$是可以成立的,就不能保证$f$是满射的。

\end{proof}

\item[12]设$f:A\rightarrow B,g:C\rightarrow D,f\subseteq g,C\subseteq A$,证明:$f=g$.

\begin{proof}
  $\forall c\in C\subseteq A,f(c)=b,g(c)=d.f\subseteq g\Rightarrow b=d$

$a\in A-C$,若$f(a)$存在,而$f\subseteq g$,矛盾。所以$A=C$。

所以$f=g$。
\end{proof}
\end{firstlayer}

\end{CJK}

\end{document}

