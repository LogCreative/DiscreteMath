% Options for packages loaded elsewhere
\PassOptionsToPackage{unicode}{hyperref}
\PassOptionsToPackage{hyphens}{url}
%
\documentclass[
]{article}
\usepackage{lmodern}
\usepackage{amssymb,amsmath}
\usepackage{ifxetex,ifluatex}
\ifnum 0\ifxetex 1\fi\ifluatex 1\fi=0 % if pdftex
  \usepackage[T1]{fontenc}
  \usepackage[utf8]{inputenc}
  \usepackage{textcomp} % provide euro and other symbols
\else % if luatex or xetex
  \usepackage{unicode-math}
  \defaultfontfeatures{Scale=MatchLowercase}
  \defaultfontfeatures[\rmfamily]{Ligatures=TeX,Scale=1}
\fi
% Use upquote if available, for straight quotes in verbatim environments
\IfFileExists{upquote.sty}{\usepackage{upquote}}{}
\IfFileExists{microtype.sty}{% use microtype if available
  \usepackage[]{microtype}
  \UseMicrotypeSet[protrusion]{basicmath} % disable protrusion for tt fonts
}{}
\makeatletter
\@ifundefined{KOMAClassName}{% if non-KOMA class
  \IfFileExists{parskip.sty}{%
    \usepackage{parskip}
  }{% else
    \setlength{\parindent}{0pt}
    \setlength{\parskip}{6pt plus 2pt minus 1pt}}
}{% if KOMA class
  \KOMAoptions{parskip=half}}
\makeatother
\usepackage{xcolor}
\IfFileExists{xurl.sty}{\usepackage{xurl}}{} % add URL line breaks if available
\IfFileExists{bookmark.sty}{\usepackage{bookmark}}{\usepackage{hyperref}}
\hypersetup{
  hidelinks,
  pdfcreator={LaTeX via pandoc}}
\urlstyle{same} % disable monospaced font for URLs
\usepackage{longtable,booktabs}
% Correct order of tables after \paragraph or \subparagraph
\usepackage{etoolbox}
\makeatletter
\patchcmd\longtable{\par}{\if@noskipsec\mbox{}\fi\par}{}{}
\makeatother
% Allow footnotes in longtable head/foot
\IfFileExists{footnotehyper.sty}{\usepackage{footnotehyper}}{\usepackage{footnote}}
\makesavenoteenv{longtable}
\setlength{\emergencystretch}{3em} % prevent overfull lines
\providecommand{\tightlist}{%
  \setlength{\itemsep}{0pt}\setlength{\parskip}{0pt}}
\setcounter{secnumdepth}{-\maxdimen} % remove section numbering

\author{}
\date{}

\begin{document}

2.由下列真值表,分别从T和F来列写出\(A\),\(B\)和\(C\)的表达式,并分别以符号\(m_i\)和\(M_i\)表示.

\begin{longtable}[]{@{}lllll@{}}
\toprule
\(P\) & \(Q\) & \(A\) & \(B\) & \(C\)\tabularnewline
\midrule
\endhead
F & F & T & T & T\tabularnewline
F & T & T & F & F\tabularnewline
T & F & T & F & F\tabularnewline
T & T & F & T & F\tabularnewline
\bottomrule
\end{longtable}

\(A=P\uparrow Q=\neg P \vee \neg Q=M_0=\neg(P\wedge Q)=\neg m_3=\vee_{0,1,2}\)

\(B=P\leftrightarrow Q=(\neg P \wedge \neg Q)\vee (P \wedge Q)=m_0\vee m_3=\vee_{0,3}=(P \vee \neg Q)\wedge (\neg P\vee Q)=M_1\wedge M_2=\wedge_{1,2}\)

\(C=P\downarrow Q=\neg P\wedge \neg Q=m_0=\neg (P\vee Q)=\neg M_3=\wedge_{0,1,2}\)

4.证明

(1)\(A→B\)与\(B^*→A^*\)同永真、同可满足

通过证明重言蕴含来同时证明两者。

若\(A\rightarrow B\)真,则\(\neg B\rightarrow \neg A\)真。(逆否命题)\\
而\(\neg A=A^{*-},\neg B=B^{*-}\),则\(B^{*-}\rightarrow A^{*-}\)真。(置换规则)\\
则假设\(A=A(P_1,P_2,\cdots,P_n)\),\(A^-=A(\neg P_1,\neg P_2,\cdots,\neg P_n)\),其中\(P_i(i=1,2,\cdots,n)\)是任意的命题,可以将\(\neg P_i\)代入后者,将得到\(A\),也就意味着\(B^*\rightarrow A^*\)真。也就是

\(A\rightarrow B\Rightarrow B^*\rightarrow A^*\)

如果\(B^*\rightarrow A^*\)真,则\((A^*)^*\rightarrow (B^*)^*\)永真。(上一个结论)\\
则\(A\rightarrow B\)真。(\((A^*)^*=A\))\\
也就是

\(B^*\rightarrow A^* \Rightarrow A\rightarrow B\)

综合两式,有

\(A\rightarrow B\Leftrightarrow B^*\rightarrow A^*\)

(i)
同永真:如果\(A→B\)永真,则根据永真蕴含的定义,\(B^*→A^*\)也将永真。反之亦然。

(ii) 同可满足:

如果\(A(a_1,a_2,\cdots,a_n),B(b_1,b_2,\cdots,b_n)\)是满足\(A→B\)真的一个条件,则根据永真蕴含的定义,\(B^*→A^*\)也将真。反之亦然。

(2)\(A\leftrightarrow B\)与\(A^*\leftrightarrow B^*\)同永真、同可满足

同样地,若\(A\leftrightarrow B\)真,则\(A=B\)(等值定理),\(\neg A=\neg B\)(同真假),\(\neg A\leftrightarrow \neg B\)真(等值定理)。

之后,根据类似的推理,\(A^{*-}\leftrightarrow B^{*-},A^{*}\leftrightarrow B^{*}\),也就是

\(A\leftrightarrow B \Rightarrow A^*\leftrightarrow B^*\)

反之,\(A^*\leftrightarrow B^*\)真,则\((A^*)^*\leftrightarrow (B^*)^*\)真,\(A\leftrightarrow B\)真,也就是

\(A^*\leftrightarrow B^* \Rightarrow A\leftrightarrow B\)

综合两个式子有

\(A\leftrightarrow B \Leftrightarrow A^*\leftrightarrow B^*\)

故两者同永真,同可满足。

5.给出下列各公式的合取范式、析取范式、主合取范式和主析取范式。并给出所有使公式为真的解释。

(4) \((P\wedge Q)\vee (\neg P\wedge Q \wedge R)\)

\((P\wedge Q)\vee (\neg P\wedge Q \wedge R)(析取范式)=(P\vee \neg P)\wedge (P \vee Q) \wedge (P \vee R) \wedge (Q\vee \neg P)\wedge (Q\vee Q)\wedge (Q\vee R)=(P\vee Q)\wedge (P\vee R)\wedge (Q\vee \neg P)\wedge Q\wedge (Q\vee R)=Q\wedge(P\vee R)(合取范式)\)

\((P\wedge Q)\vee (\neg P\wedge Q \wedge R)=(P\wedge Q\wedge (R\vee \neg R))\vee (\neg P\wedge Q \wedge R)=(P\wedge Q\wedge R)\vee(P\wedge Q\wedge \neg R)\vee (\neg P\wedge Q \wedge R)=m_7\vee m_6\vee m_3=\vee_{3,6,7}(主析取范式)=\wedge_{\{0,1,2,4,5\}\text{补}}=\wedge_{2,3,5,6,7}(主合取范式)\)

使公式为真的解释:

\begin{longtable}[]{@{}llll@{}}
\toprule
\(P\) & \(Q\) & \(R\) & \(m_i\)\tabularnewline
\midrule
\endhead
F & T & T & \(m_3\)\tabularnewline
T & T & F & \(m_6\)\tabularnewline
T & T & T & \(m_7\)\tabularnewline
\bottomrule
\end{longtable}

(8)
\((P\rightarrow Q)\vee ((Q\wedge P)\leftrightarrow (Q\leftrightarrow \neg P))\)

\((P\rightarrow Q)\vee ((Q\wedge P)\leftrightarrow (Q\leftrightarrow \neg P))=(P\rightarrow Q)\vee ((Q\wedge P)\leftrightarrow (\neg Q \wedge \neg \neg P)\vee (Q \wedge \neg P))=(\neg P \vee Q)\vee [(Q\wedge P)\leftrightarrow (\neg Q \wedge P)\vee (Q \wedge \neg P)]=(\neg P \vee Q)\vee\{[(Q\wedge P)\wedge ((\neg Q \wedge P)\vee (Q \wedge \neg P))]\vee [\neg (Q\wedge P)\wedge \neg ((\neg Q \wedge P)\vee (Q \wedge \neg P))] \}=(\neg P \vee Q)\vee\{[((Q\wedge P)\wedge (\neg Q \wedge P))\vee ((Q\wedge P)\wedge(Q \wedge \neg P))]\vee [\neg (Q\wedge P)\wedge (\neg(\neg Q \wedge P)\wedge \neg(Q \wedge \neg P))] \}=(\neg P \vee Q)\vee\{\text{F}\vee[(\neg Q\vee \neg P)\wedge (( Q \vee \neg P)\wedge ( P\vee \neg Q))] \}=(\neg P \vee Q)\vee\{[(\neg Q\vee \neg P)\wedge ( P\vee \neg Q)]\wedge ( \neg P \vee Q) \}=\neg P \vee Q(析取范式、合取范式)=M_1=\wedge_1(主合取范式)=\vee_{\{0,2,3\}\text{补}}=\vee_{0,1,3}(主析取范式)\)

使公式为真的解释:

\begin{longtable}[]{@{}lll@{}}
\toprule
\(P\) & \(Q\) & \(m_i\)\tabularnewline
\midrule
\endhead
F & F & \(m_0\)\tabularnewline
F & T & \(m_1\)\tabularnewline
T & T & \(m_3\)\tabularnewline
\bottomrule
\end{longtable}

\end{document}
