%md->tex->template
\documentclass[12pt]{article}
\usepackage{CJKutf8}
\usepackage{amsmath}
\usepackage{geometry}
\usepackage{fancyhdr}
\usepackage{longtable,booktabs}
\usepackage{enumerate}
\usepackage{enumitem}
\usepackage{amsthm}
\setlist[enumerate,1]{font=\bfseries}
\geometry{left=3.0cm,right=2.0cm,top=3.0cm,bottom=3.0cm}

\newenvironment{firstlayer}%
{\begin{list}{}{\renewcommand{\makelabel}[1]{\textbf{##1}.\hfil}
}}
{\end{list}}
\newenvironment{secondlayer}%
{\begin{list}{}{\renewcommand{\makelabel}[1]{(##1)\hfil}
}}
{\end{list}}

\renewcommand{\proofname}{\textbf{证明}}

\title{第 2 次作业}
\author{李子龙 518070910095}
\date{March 15, 2020}
\begin{document}

\begin{CJK}{UTF8}{gbsn}

\maketitle

\begin{firstlayer}
\item[1] 证明下列等值公式。
    \begin{secondlayer}
    \item[1] $P\rightarrow (Q\wedge R)=(P\rightarrow Q)\wedge (P\rightarrow R)$
    \begin{proof}
      \begin{align*}
        P\rightarrow (Q\wedge R)&= \neg P\vee (Q\wedge R) &&\text{($\rightarrow$的等值公式)} \\
        &=(\neg P \vee Q)\wedge(\neg P \vee R) &&\text{(分配律)} \\
        &=(P\rightarrow Q)\wedge (P\rightarrow R) &&\text{($\rightarrow$的等值公式)}
      \end{align*}
    \end{proof}
    \item[3] $((P\rightarrow \neg Q)\rightarrow (Q\rightarrow \neg P))\wedge R=R$
    \begin{proof}
      \begin{align*}
        ((P\rightarrow \neg Q)\rightarrow (Q\rightarrow \neg P))\wedge R&=((P\rightarrow \neg Q)\rightarrow (\neg \neg P\rightarrow \neg Q))\wedge R &&\text{(逆否定理)} \\
        &=((P\rightarrow \neg Q)\rightarrow (P\rightarrow \neg Q))\wedge R &&\text{(双重否定)}\\
        &=\text{T}\wedge R &&\text{(等幂律)}\\
        &=R &&\text{(同一律)}
      \end{align*}
    \end{proof}
    \item[5] $P\rightarrow (Q\rightarrow R) = (P\wedge Q)\rightarrow R$
    \begin{proof}
      \begin{align*}
        P\rightarrow (Q\rightarrow R)&=P\rightarrow (\neg Q\vee R)&&\text{($\rightarrow$的等值公式)} \\
        & =\neg P\vee (\neg Q\vee R) &&\text{($\rightarrow$的等值公式)} \\
        & =(\neg P\vee \neg Q)\vee R &&\text{(结合律)}\\
        & =\neg (P\wedge Q)\vee R &&\text{(摩根律)}\\
        & =(P\wedge Q)\rightarrow R &&\text{($\rightarrow$的等值公式)}
      \end{align*}
    \end{proof}
    \item[6] $\neg (P\leftrightarrow Q)=(P\wedge \neg Q)\vee (\neg P \wedge Q)$
    \begin{proof}
      \begin{align*}
        \neg (P\leftrightarrow Q)&=\neg ((P\rightarrow Q)\wedge (Q\rightarrow P)) &&\text{($\leftrightarrow$定义)}\\
        & =\neg(P\rightarrow Q)\vee \neg(Q\rightarrow P) &&\text{(摩根律)} \\
        & =\neg(\neg P \vee Q)\vee \neg(\neg Q \vee P) &&\text{($\rightarrow$的等值公式)}\\
        &=(\neg \neg P \wedge \neg Q)\vee (\neg \neg Q \wedge \neg P) &&\text{(摩根律)}\\
        &=(P \wedge \neg Q)\vee (Q \wedge \neg P)&&\text{(双重否定)}  \\
        &=(P\wedge \neg Q)\vee (\neg P \wedge Q) &&\text{(交换律)}
      \end{align*}
    \end{proof}
    \end{secondlayer}
\item[3] 用$\uparrow$和$\downarrow$分别表示出$\neg$, $\wedge$, $\vee$, $\rightarrow$和 $\leftrightarrow$。
\begin{proof}
  令$P,Q$为命题变元。
  \begin{subequations}
    \begin{align}
      \neg P&=\neg P\vee \neg P && \text{(等幂律)} \nonumber \\
      &=P\uparrow P && \text{($\uparrow$定义)} \label{neg1}\\
      \neg P&=\neg P\wedge \neg P &&\text{(等幂律)}\nonumber \\
      &=P\downarrow P &&\text{($\downarrow$定义)} \label{neg2}
    \end{align}
  \end{subequations}
  \begin{align}
    P\wedge Q&=\neg \neg P \wedge \neg \neg Q&& \text{(双重否定)} \nonumber \\
    &=\neg P \downarrow \neg Q &&\text{($\downarrow$ 定义)} \nonumber \\
    &=(P\uparrow P)\downarrow (Q\uparrow Q)&&\text{(公式 (\ref{neg1}))} \label{wedge1}
  \end{align}
  \begin{align}
    P \vee Q&=\neg \neg P \vee \neg \neg Q &&\text{(双重否定)} \nonumber \\
    &=\neg P\uparrow \neg Q &&\text{($\uparrow$定义)} \nonumber \\
    &=(P \downarrow P)\uparrow (Q \downarrow Q)  &&\text{(公式 (\ref{neg2}))} \label{vee1}
  \end{align}
  \begin{align}
    P \rightarrow Q&=\neg P\vee Q&&\text{($\rightarrow$的等值公式)} \nonumber \\
    &=(P\uparrow P)\vee Q&&\text{(公式 (\ref{neg1}))} \nonumber \\
    &=((P\uparrow P)\downarrow (P\uparrow P))\uparrow (Q\downarrow Q) &&\text{(公式 (\ref{vee1}))} \label{rightarrow1}
  \end{align}
  \begin{align*}
    P\leftrightarrow Q=&(P\rightarrow Q)\wedge(Q\rightarrow P)&&\text{($\leftrightarrow$定义)} \nonumber \\
    =&(((P\uparrow P)\downarrow (P\uparrow P))\uparrow (Q\downarrow Q)) 
    \wedge (((Q\uparrow Q)\downarrow (Q\uparrow Q))\uparrow (P\downarrow P))&&\text{(公式 (\ref{rightarrow1}))} \nonumber \\
    =&((((P\uparrow P)\downarrow (P\uparrow P))\uparrow (Q\downarrow Q)) \uparrow (((P\uparrow P)\downarrow (P\uparrow P))\uparrow (Q\downarrow Q)))) \nonumber \\
    &\downarrow ((((Q\uparrow Q)\downarrow (Q\uparrow Q))\uparrow (P\downarrow P))\uparrow (((Q\uparrow Q)\downarrow (Q\uparrow Q))\uparrow (P\downarrow P))) &&\text{(公式 (\ref{wedge1}))}
  \end{align*}
\end{proof}
\end{firstlayer}

\end{CJK}

\end{document}

