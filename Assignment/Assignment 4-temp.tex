%md->tex->template
\documentclass[12pt]{article}
\usepackage{CJKutf8}
\usepackage{amsmath}
\usepackage{geometry}
\usepackage{fancyhdr}
\usepackage{longtable,booktabs}
\usepackage{enumerate}
\usepackage{enumitem}
\usepackage{amsthm}
\usepackage{amssymb}
\setlist[enumerate,1]{font=\bfseries}
\geometry{left=3.0cm,right=2.0cm,top=3.0cm,bottom=3.0cm}

\newenvironment{firstlayer}%
{\begin{list}{}{\renewcommand{\makelabel}[1]{\textbf{##1}.\hfil}
}}
{\end{list}}
\newenvironment{secondlayer}%
{\begin{list}{}{\renewcommand{\makelabel}[1]{(##1)\hfil}
}}
{\end{list}}

\renewcommand{\proofname}{\textbf{证明}}

\title{第 4 次作业}
\author{Log Creative}
\date{March 29, 2020}
\begin{document}

\begin{CJK}{UTF8}{gbsn}

\maketitle

\section*{第二章}

\begin{firstlayer}
  \item[7] 判断下列推理式是否正确?
  \begin{secondlayer}
    \item[10] \(((P\wedge Q)\rightarrow R)\wedge ( ( P\vee Q)\rightarrow \neg R)\Rightarrow P\wedge Q\wedge R\)

    \textbf{解.} 错误。理由:

\begin{align*}
  ((P\wedge Q)\rightarrow R)&\wedge (( P\vee Q)\rightarrow \neg R)\rightarrow P\wedge Q\wedge R \\
  =&\neg ((\neg (P\wedge Q)\vee R)\wedge (\neg ( P\vee Q)\vee \neg R))\vee (P\wedge Q\wedge R) \\
  =&((P\wedge Q)\wedge \neg R)\vee((P\vee Q)\wedge R)\vee(P\wedge Q\wedge R) \\
  =&(P\wedge Q\wedge \neg R)\vee(P\wedge R)\vee(Q\wedge R)\vee(P\wedge Q\wedge R) \\
  =&((P\wedge Q)\wedge(\neg R\vee R))\vee(P\wedge R)\vee(Q\wedge R) \\
  =&(P\wedge Q)\vee (P\wedge R)\vee(Q\wedge R)
\end{align*}

当\(P=Q=\text{F}\)时,该式为\(\text{F}\)。故根据\(P\rightarrow Q\)为真与\(P\Rightarrow Q\)等价的关系可以得到推理式错误。

    \item[11] \(P \rightarrow Q \Rightarrow ( P \rightarrow R) \rightarrow (Q \rightarrow R)\)

    \textbf{解.} 错误。理由:

\begin{align*}
  (P \rightarrow Q) \rightarrow & (( P \rightarrow R) \rightarrow (Q \rightarrow R)) \\
  =&\neg (\neg P\vee Q)\vee(\neg(\neg P\vee R)\vee (\neg Q\vee R))\\
  =&(P\wedge \neg Q)\vee(P\wedge \neg R)\vee(\neg Q\vee R)\\
  =&\neg Q\vee (P\wedge \neg R)\vee R \\
  =&\neg Q\vee P\vee R\neq \text{T}
\end{align*}

  \end{secondlayer}

  \item[8] 使用推理规则证明
  \begin{secondlayer}
    \item[4] \(P \vee Q\rightarrow R\wedge S, S\vee E \rightarrow U \Rightarrow P \rightarrow U\)
    \begin{proof}
        \begin{subequations}
        \begin{align}
        &P \vee Q\rightarrow R\wedge S && \text{(前提引入)} \\
        &P && \text{(附加前提引入)} \label{a1} \\
        &R\wedge S && \text{(分离)}\\
        &S && \text{(合取)}\\
        &S\vee E \rightarrow U && \text{(前提引入)} \\
        &U && \text{(分离)} \label{b1} \\
        &P\rightarrow U && \text{((\ref{a1})和(\ref{b1}) 条件证明规则)}
        \end{align}

        \end{subequations}
    \end{proof}
    \item[5] \(\neg R\vee S, S\rightarrow Q, \neg Q\Rightarrow Q \leftrightarrow R\)
    \begin{proof}
      \begin{subequations}
      \begin{align}
        &S\rightarrow Q  && \text{(前提引入)} \\
        &\neg Q\rightarrow \neg S && \text{(置换)}   \\
        &\neg Q && \text{(前提引入)} \label{a} \\
        &\neg S  && \text{(分离)}    \\
        &\neg R\vee S && \text{(前提引入)}\\
        &\neg S \rightarrow \neg R  && \text{(置换)}  \\
        &\neg R && \text{(分离)} \label{b} \\
        &Q\leftrightarrow R && \text{((\ref{a})和(\ref{b}) 条件证明规则)}
      \end{align}

      \end{subequations}
    \end{proof}
    \item[6] \(\neg Q\vee S, ( E\rightarrow \neg U)\rightarrow \neg S\Rightarrow Q\rightarrow E\)
    \begin{proof}
        \begin{subequations}
            \begin{align}
              &\neg Q\vee S && \text{(前提引入)} \\
          &Q\rightarrow S   && \text{(置换)}       \\
          &Q && \text{(附加前提引入)} \label{a2} \\
          &S && \text{(分离)} \\
          &( E\rightarrow \neg U)\rightarrow \neg S  && \text{(前提引入)} \\
          &S\rightarrow \neg (E\rightarrow \neg U)  && \text{(置换)}  \\
          &\neg (E\rightarrow \neg U) && \text{(分离)} \\
          &E\wedge U && \text{(置换)} \\
          &E && \text{(合取)} \label{b2} \\
          &Q\rightarrow E && \text{((\ref{a2})和(\ref{b2}) 条件证明规则)}
            \end{align}

        \end{subequations}
    \end{proof}
    \item[补充] \(P\rightarrow (Q\rightarrow R),Q\rightarrow (R\rightarrow S)\Rightarrow P\rightarrow (R\rightarrow S)\)
    \begin{proof}
        \begin{subequations}
        \begin{align}
          &P\rightarrow (Q\rightarrow R)  && \text{(前提引入)}    \\
            &P   && \text{(附加前提引入)} \label{a3} \\
            &Q\rightarrow R  && \text{(分离)} \\
            &Q\rightarrow (R\rightarrow S)  && \text{(前提引入)} \\
            &(Q\rightarrow R)\rightarrow (R\rightarrow S)   && \text{(置换)} \\
            &R\rightarrow S && \text{(分离)} \label{b3} \\
            &P\rightarrow (R\rightarrow S) && \text{((\ref{a3})和(\ref{b3}) 条件证明规则)}
        \end{align}

        \end{subequations}
    \end{proof}
  \end{secondlayer}
  \item[9] 证明下列推理关系:
  \begin{secondlayer}
    \item[1] 在大城市球赛中. 如果北京队第三, 那么如果上海队第二, 那么天津队第四. 沈阳队不是第一或北京队第三. 上海队第二. 从而知, 如果沈阳队第一, 那么天津队第四.
    \begin{proof}
      令:\(A_1=\)沈阳队第一,\(A_2=\)上海队第二,\(A_3=\)北京队第三,\(A_4=\)天津队第四。则原命题可以陈述为

      $$A_3\rightarrow (A_2\rightarrow A_4),\neg A_1\vee A_3,A_2\Rightarrow A_1\rightarrow A_4$$
      \begin{subequations}
      \begin{align}
        &\neg A_1\vee A_3    && \text{(前提引入)}   \\
        &A_1\rightarrow A_3  && \text{(置换)}  \\
        &A_1   && \text{(附加前提引入)} \label{a4}  \\
        &A_3   && \text{(分离)}  \\
        &A_3\rightarrow (A_2\rightarrow A_4)  && \text{(前提引入)}   \\
        &A_2\rightarrow A_4 && \text{(分离)}      \\
        &A_2     && \text{(前提引入)}  \\
        &A_4   && \text{(分离)} \label{b4}  \\
        &A_1\rightarrow A_4  && \text{((\ref{a4})和(\ref{b4}) 条件证明规则)}
      \end{align}

      \end{subequations}
    \end{proof}
  \end{secondlayer}
  \item[12] 利用归结法证明
  \begin{secondlayer}
    \item[1] \(( P \vee Q) \wedge ( P \rightarrow R) \wedge (Q\rightarrow R)\Rightarrow R\)
    \begin{proof}
        \begin{align*}
          ( P \vee Q) &\wedge ( P \rightarrow R) \wedge (Q\rightarrow R)\wedge \neg R \\
          =&( P \vee Q) \wedge ( \neg P \vee R) \wedge (\neg Q\vee R)\wedge \neg R
        \end{align*}

        建立子句集\(S=\{ P \vee Q, \neg P \vee R, \neg Q\vee R, \neg R \}\)
        \begin{subequations}
        \begin{align}
          &P \vee Q  \label{1a} \\
            &\neg P \vee R \label{1b} \\
            &\neg Q\vee R \label{1c} \\
            &\neg R \label{1d} \\
            &Q\vee R  && \text{((\ref{1a})和(\ref{1b}) 归并)} \label{1e} \\
            &R && \text{((\ref{1c})和(\ref{1e}) 归并)} \label{1f} \\
            & \square && \text{((\ref{1e})和(\ref{1f}) 归并)}
        \end{align}

        \end{subequations}
    \end{proof}
  \end{secondlayer}
\end{firstlayer}

\section*{第四章}
\begin{firstlayer}
  \item[1] 判断下列各式是否合式公式
  \begin{secondlayer}
    \item[1] \(P ( x ) \vee(\forall x )Q( x )\)

    \textbf{解.} 不是。同一变量两边辖域不同。
    \item[2] \(( \forall x ) ( P ( x ) \wedge Q( x ) )\)

    \textbf{解.} 是。
    \item[4] \(( \exists x ) P ( y , z )\)

    \textbf{解.} 是。
    \item[6] \(( \forall x ) ( P ( x ) \wedge R( x ) ) \rightarrow ( ( \forall x ) P ( x ) \wedge Q( x ) )\)

    \textbf{解.} 不是。右侧的同一变量两侧辖域不同。

    \item[8] \(( \exists x ) ( ( \forall y) P ( y) \rightarrow  Q( x , y ) )\)

    \textbf{解.} 不是。\(( \forall y) P ( y) \rightarrow  Q( x , y )\)内\(y\)的辖域不同。

    \item[9] \(( \exists x ) ( \exists y ) ( P ( x , y , z ) \rightarrow S( u , v) )\)

    \textbf{解.} 是。
  \end{secondlayer}
  \item[2] 作如何的具体设定下列公式方为命题

  \begin{secondlayer}
    \item[3] \(( \forall x ) ( \exists y ) P ( x , f ( y , a ) ) \wedge Q( z)\)

\textbf{解.} 当且仅当\(x\),\(y\),\(a\),\(z\)取为常数,\(f\)是常函数,\(P,Q\)为谓词常量。
  \end{secondlayer}
  \item[3] 指出下列公式中的自由变元和约束变元, 并指出各量词的辖域

  \begin{secondlayer}
    \item[2] \(( \forall x ) ( P ( x ) \wedge ( \exists y )Q( y ) ) \wedge ( ( \forall x ) P ( x ) \rightarrow Q( z) )\)

    \textbf{解.} \(z\)是自由变元,\(x,y\)是约束变元。

\(( \forall x ) ( P ( x ) \wedge ( \exists y )Q( y ) )\),\(P ( x ) \wedge ( \exists y )Q( y )\)是\(x\)的辖域。

\(( \exists y )Q( y )\),\(Q( y )\)是\(y\)的辖域。

\(( \forall x ) P ( x )\),\(P ( x )\)是\(x\)的辖域。

    \item[3] \(( \forall x ) ( P ( x ) \leftrightarrow Q( x ) ) \wedge ( \exists y ) R( y ) \wedge S( z)\)

    \textbf{解.} \(z\)是自由变元,\(x,y\)是约束变元。

\(( \forall x ) ( P ( x ) \leftrightarrow Q( x ) )\),\(P ( x ) \leftrightarrow Q( x )\)是\(x\)的辖域。

\(( \exists y ) R( y )\),\(R( y )\)是\(y\)的辖域。
  \end{secondlayer}
  \item[4] 求下列各式的真值

  \begin{secondlayer}
    \item[2] \(( \exists x ) ( P\rightarrow Q( x ) ) \wedge R( a )\).论域为\(\{- 2, 1, 2, 3, 5, 6\}\),
\(P\) 表 \(2> 1\), \(Q( x )\)表 \(x \leq 3\), \(R( x )\)
表\(x > 5,a= 3\).

    \textbf{解.}
\(P=\text{T}\)

\((\exists x)(P\rightarrow Q( x ))=\text{T}\)(因为论域中有满足\(x \leq 3\)的数字,\(\text{T}\rightarrow \text{T}=\text{T}\))

\(R(a)=\text{F}\)(因为\(3 \ngeq 5\))

\(( \exists x ) ( P\rightarrow Q( x ) ) \wedge R( a )=\text{F}\)
  \end{secondlayer}
\end{firstlayer}


\end{CJK}

\end{document}

